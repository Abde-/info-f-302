\documentclass[a4paper,11pt]{article}
\usepackage[T1]{fontenc}
\usepackage[utf8]{inputenc}
\usepackage{lmodern}
\usepackage{graphicx} % to include pictures
\usepackage{hyperref}
\usepackage{tcolorbox}
\usepackage{amsmath}
\usepackage{breqn}
\usepackage{amsfonts}
\usepackage[top=2cm, bottom=2cm, left=2.5cm, right=2.5cm]{geometry}

% Fancy layout
\usepackage{fancyhdr}
\pagestyle{fancy}
% Footer
\fancyfoot[C]{}
\fancyfoot[R]{Page \thepage}
% Ligne
\renewcommand{\headrulewidth}{0.4pt}
\renewcommand{\footrulewidth}{0.4pt}

\begin{document}
\section{Problème 4 (minimisation de cavaliers)}
\subsection{Notations}
\begin{itemize}
\item n x n = dimension de l'échiquier
\end{itemize}

\subsection{Variables}
\begin{equation}
  \begin{split}
    X = \{a_1, \dots, a_{n*n} \}
  \end{split}
\end{equation}

\subsection{Domaine}
$$D_a = \{ 0, 1 \}$$
$$D_m = \mathbb{N}$$

\subsubsection{Contraintes}
\begin{itemize}
\item Contrainte tel qu'il faut minimiser le nombre de cavaliers dans l'échiquier.
  $$ c_1 = minimize(\sum_{k=1}^{n*n} a_k) $$
  
\item[•] Contrainte tel que chaque case doit être dominé ou occupé par un cavalier.
  
  $ c_2 = (a_1, \dots, a_{n*n} | \forall \hspace{0.1 cm}1 \leq i \leq n,  \forall  \hspace{0.1 cm}1 \leq j \leq n, \exists k \in \{2,-2\}, \exists l \in \{1,-1\}, \\
   a_{(i+k)*n +j +l}=1 \vee a_{(i+l)*n +j +k}=1 \vee a_{(i*n +j)} = 1) $ 
\end{itemize}

\end{document}