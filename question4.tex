\documentclass[a4paper,11pt]{article}
\usepackage[T1]{fontenc}
\usepackage[utf8]{inputenc}
\usepackage{lmodern}
\usepackage{graphicx} % to include pictures
\usepackage{hyperref}
\usepackage{tcolorbox}
\usepackage{amsmath}
\usepackage{breqn}
\usepackage{amsfonts}
\usepackage[top=2cm, bottom=2cm, left=2.5cm, right=2.5cm]{geometry}

% Fancy layout
\usepackage{fancyhdr}
\pagestyle{fancy}
% Footer
\fancyfoot[C]{}
\fancyfoot[R]{Page \thepage}
% Ligne
\renewcommand{\headrulewidth}{0.4pt}
\renewcommand{\footrulewidth}{0.4pt}

\begin{document}
\section{Problème 4 (minimisation de cavaliers)}
\subsection{Notations}
\begin{itemize}
\item n x n = dimension de l'échiquier
\item y = $a_{1,1},\ldots, a_{1,n},a_{2,1}, \ldots,a_{2,n}, \ldots, a_{n,n} $
\item z = $v_{1,1},\ldots, v_{1,n},v_{2,1}, \ldots,v_{2,n}, \ldots, v_{n,n} $
\end{itemize}

\subsection{Variables}
\begin{equation}
  \begin{split}
    X = \{a_{1,1},\ldots, a_{1,n},a_{2,1}, \ldots,a_{2,n}, \ldots, a_{n,n} \}
  \end{split}
\end{equation}

\subsection{Domaine}
$$D = \{ 0, 1 \}$$

\subsection{Contraintes}
L'ensemble des contraintes est défini tel que chaque case doit être dominée ou occupée par un chevalier. 
  $$ C = (y, \{z \hspace{0.1cm} | \hspace{0.1cm} \forall \hspace{0.1cm} 1 \leq i \leq n, \forall \hspace{0.1cm} 1 \leq j \leq n, \exists k \in \{2,-2\}, \exists l \in \{1,-1\}, a_{i+k,j+l}=1 \hspace{0.1cm} \vee \hspace{0.1cm} a_{i+l,j+k}=1 \hspace{0.1cm} \vee \hspace{0.1cm} a_{i,j)} = 1\}) $$ 

\subsection{Fonction objective}

La fonction objective de ce CSP est défini tel que: [expliquer]
  $$ \phi = minimize(\sum_{j=1}^{n} \sum_{k}^n a_{j,k}) $$
  
\end{document}