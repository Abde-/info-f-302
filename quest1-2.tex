\documentclass[a4paper,11pt]{article}
\usepackage[T1]{fontenc}
\usepackage[utf8]{inputenc}
\usepackage{lmodern}
\usepackage{graphicx} % to include pictures
\usepackage{hyperref}
\usepackage{tcolorbox}
\usepackage{amsmath}
\usepackage{breqn}
\usepackage{amsfonts}
\usepackage[top=2cm, bottom=2cm, left=2.5cm, right=2.5cm]{geometry}

% Fancy layout
\usepackage{fancyhdr}
\pagestyle{fancy}
% Footer
\fancyfoot[C]{}
\fancyfoot[R]{Page \thepage}
% Ligne
\renewcommand{\headrulewidth}{0.4pt}
\renewcommand{\footrulewidth}{0.4pt}

\begin{document}
\section{Problème 1 (échiquier)}
\subsection{Notations}
\begin{itemize}
\item n x n = dimension de l'échiquier
\item $k_{1}$ = nombre de tours
\item $k_{2}$ = nombre de fous
\item $k_{3}$ = nombre de cavaliers

\item m = ($k_{1}$ + $k_{2}$ + $k_{3}$)
\item y = ($t_{1}, t'_{1}, ..., t_{k_{1}}, t'_{k_{1}}, f_{1}, f'_{1}, ..., f_{k_{2}}, f'_{k_{2}}, h_{1}, h'_{1}, ..., h_{k_{3}}, h'_{k_{3}}$)
\end{itemize}

\subsection{Question 1}
\subsubsection{Variables}
\begin{equation}
  \begin{split}
    X = \{ &t_{i} : i \epsilon  \{1, ..., k_{1} \}, f_{i} : i \epsilon  \{1, ..., k_{2} \}, h_{i} : i \epsilon  \{1, ..., k_{3} \}, \\
    &t'_{i} : i \epsilon  \{1, ..., k_{1} \}, f'_{i} : i \epsilon  \{1, ..., k_{2} \}, h'_{i} : i \epsilon  \{1, ..., k_{3} \} \}
  \end{split}
\end{equation}

\subsubsection{Domaine}
$$D = \{1, ..., n \}$$

\subsubsection{Contraintes}
\begin{itemize}
\item Contrainte pour indiquer que 2 pièces peuvent pas se trouver dans la même case.
   $$c_{1} = ( y, \{ (v_{1}, ..., v_{2m}) \epsilon  D^{2m} \mid \forall_{i,j} 1\leq i \neq j \leq m \rightarrow (v_{2i} \neq v_{2j}) \vee  (v_{2i+1} \neq v_{2j+1}) \} ) $$
\item Contrainte pour la portée des tours 
 $$ c_{2} = ( y, \{ (v_{1}, ..., v_{2m}) \epsilon  D^{2m} \mid \forall_{i,j} 1\leq i \leq k_{1}, 1\leq j \leq m  \rightarrow (v_{2i} \neq v_{2j}) \wedge  (v_{2i+1} \neq v_{2j+1}) \} ) $$

\item Contrainte pour la portée des fous 
  $$c_{3} = ( y, \{ (v_{1}, ..., v_{2m}) \epsilon  D^{2m} \mid \forall_{i,j} k_{1}+1\leq i \leq k_{1}+k_{2}, 1\leq j \leq m \rightarrow$$
  $$[ (v_{2i} \neq v_{2j}+k) \wedge  (v_{2i+1} \neq v_{2j+1}+k) \wedge  (v_{2i} \neq v_{2j}-k) \wedge  (v_{2i+1} \neq v_{2j+1}+k) ],$$
  $$k \epsilon \mathbb{Z}_{0} \} ) $$

\item Contrainte pour la portée des cavalier 
  $$ c_{4} = ( y, \{ (v_{1}, ..., v_{2m}) \epsilon  D^{2m} \mid \forall_{i,j} k_{1}+k_{2}+1\leq i \leq k_{1}+k_{2}+k_{3}, 1\leq j \leq m \rightarrow$$
  $$[(v_{2i} \neq v_{2j}+k) \wedge  (v_{2i+1} \neq v_{2j+1}+l) \wedge  (v_{2i} \neq v_{2j}+l) \wedge  (v_{2i+1} \neq v_{2j+1}+k) ],$$
  $$k \epsilon \{-2, 2\}, l \epsilon \{-1, 1\} \} )$$

\end{itemize}

\subsection{Question 2}

\subsubsection{Contraintes}
\begin{itemize}
\item Contrainte pour indiquer que 2 pièces peuvent pas se trouver dans la même case.
   $$c_{1} = ( y, \{ (v_{1}, ..., v_{2m}) \epsilon  D^{2m} \mid \forall_{i,j} 1\leq i \neq j \leq m \rightarrow (v_{2i} \neq v_{2j}) \vee  (v_{2i+1} \neq v_{2j+1}) \} )$$ 

  
 \item Contrainte pour la portée des tours
   $$ c_{2} = ( y, \{ (v_{1}, ..., v_{2m}) \epsilon  D^{2m} \mid (\forall_{i} 1\leq i \leq k_{1} \rightarrow \exists j \epsilon (1 \leq j \leq m) \mid (v_{2i} = v_{2j}) \vee  (v_{2i+1} = v_{2j+1}) )\} ) $$

 \item Contrainte pour la portée des tours
   $$ c_{3} = ( y, \{ (v_{1}, ..., v_{2m}) \epsilon  D^{2m} \mid (\forall_{i} k_{1}+1\leq i \leq k_{1}+k_{2} \rightarrow$$
   $$\exists j \epsilon (1 \leq j \leq m) \mid [ (v_{2i} = v_{2j}+k) \wedge  (v_{2i+1} = v_{2j+1}+k) \vee  (v_{2i} = v_{2j}-k) \wedge  (v_{2i+1} = v_{2j+1}+k) ]),$$
   $$k \epsilon \mathbb{Z}_{0} \} ) $$

 \item Contrainte pour la portée des tours
   $$ c_{4} = ( y, \{ (v_{1}, ..., v_{2m}) \epsilon  D^{2m} \mid (\forall_{i} k_{1}+k_{2}+1\leq i \leq k_{1}+k_{2}+k_{3} \rightarrow$$
   $$\exists j \epsilon (1 \leq j \leq m) \mid [ (v_{2i} = v_{2j}+k) \wedge  (v_{2i+1} = v_{2j+1}+l) \vee  (v_{2i} = v_{2j}+l) \wedge  (v_{2i+1} = v_{2j+1}+k) ]),$$
   $$k \epsilon \{-2, 2\}, l \epsilon \{-1, 1\} \} ) $$

\end{itemize}

\end{document}