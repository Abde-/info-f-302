\documentclass[a4paper,11pt]{article}
\usepackage[T1]{fontenc}
\usepackage[utf8]{inputenc}
\usepackage{lmodern}
\usepackage{graphicx} % to include pictures
\usepackage{hyperref}
\usepackage{tcolorbox}
\usepackage{amsmath}
\usepackage{breqn}
\usepackage{amsfonts}
\usepackage[top=2cm, bottom=2cm, left=2.5cm, right=2.5cm]{geometry}

% Fancy layout
\usepackage{fancyhdr}
\pagestyle{fancy}
% Footer
\fancyfoot[C]{}
\fancyfoot[R]{Page \thepage}
% Ligne
\renewcommand{\headrulewidth}{0.4pt}
\renewcommand{\footrulewidth}{0.4pt}

\begin{document}
\section{Problème 1 (échiquier)}
\subsection{Notations}
\begin{itemize}
\item n x n = dimension de l'échiquier
\item $k_{1}$ = nombre de tours
\item $k_{2}$ = nombre de fous
\item $k_{3}$ = nombre de cavaliers

\item m = ($k_{1}$ + $k_{2}$ + $k_{3}$)
\item y = ($t_{1}, t'_{1}, ..., t_{k_{1}}, t'_{k_{1}}, f_{1}, f'_{1}, ..., f_{k_{2}}, f'_{k_{2}}, h_{1}, h'_{1}, ..., h_{k_{3}}, h'_{k_{3}}$)
\end{itemize}

\subsection{Variables}
\begin{equation}
  \begin{split}
    X = \{ &t_{i} : i \in  \{1, ..., k_{1} \}, f_{i} : i \in  \{1, ..., k_{2} \}, h_{i} : i \in  \{1, ..., k_{3} \}, \\
    &t'_{i} : i \in  \{1, ..., k_{1} \}, f'_{i} : i \in  \{1, ..., k_{2} \}, h'_{i} : i \in  \{1, ..., k_{3} \} \}
  \end{split}
\end{equation}

\subsection{Domaine}
$$D = \{1, ..., n \}$$

\subsection{Question 1}
\subsubsection{Contraintes}
\begin{itemize}
\item Contrainte pour indiquer que 2 pièces ne peuvent pas se trouver dans la même case.
   $$c_{1} = ( y, \{ (v_{1}, ..., v_{2m}) \in  D^{2m} \mid \forall \hspace{0.1cm}  1\leq i \neq j \leq m, (v_{2i-1} \neq v_{2j-1}) \vee  (v_{2i} \neq v_{2j}) \} ) $$
\item Contrainte pour la portée des tours 
 $$ c_{2} = ( y, \{ (v_{1}, ..., v_{2m}) \in  D^{2m} \mid \forall \hspace{0.1cm} 1\leq i \leq k_{1}, \forall \hspace{0.1cm} 1\leq j \leq m, (v_{2i-1} \neq v_{2j-1}) \wedge  (v_{2i} \neq v_{2j}) \} ) $$

\item Contrainte pour la portée des fous 
  $$c_{3} = ( y, \{ (v_{1}, ..., v_{2m}) \in  D^{2m} \mid \forall \hspace{0.1cm}  k_{1}+1\leq i \leq k_{1}+k_{2}, \forall \hspace{0.1cm} 1\leq j \leq m, \forall \hspace{0.1cm} k \in \{-n, n \}, $$
  $$[ ((v_{2i-1} \neq v_{2j-1}+k) \vee  (v_{2i} \neq v_{2j}+k)) \wedge  ((v_{2i-1} \neq v_{2j-1}-k) \vee  (v_{2i} \neq v_{2j}+k)) ] \})$$

\item Contrainte pour la portée des cavalier 
  $$ c_{4} = ( y, \{ (v_{1}, ..., v_{2m}) \in  D^{2m} \mid \forall \hspace{0.1cm} k_{1}+k_{2}+1\leq i \leq m, \forall \hspace{0.1cm} 1\leq j \leq m, \forall \hspace{0.1cm} k \in \{-2, 2\}, \forall \hspace{0.1cm}l \in \{-1, 1\}, $$
  $$[((v_{2i-1} \neq v_{2j-1}+k) \vee  (v_{2i} \neq v_{2j}+l)) \wedge ((v_{2i-1} \neq v_{2j-1}+l) \vee  (v_{2i} \neq v_{2j}+k)) ] \})$$

\end{itemize}

\newpage 
\subsection{Question 2}

\subsubsection{Contraintes}
\begin{itemize}
\item Contrainte pour indiquer que 2 pièces ne peuvent pas se trouver dans la même case.
   $$c_{1} = ( y, \{ (v_{1}, ..., v_{2m}) \in  D^{2m} \mid \forall \hspace{0.1cm} 1\leq i \neq j \leq m, (v_{2i-1} \neq v_{2j-1}) \vee  (v_{2i} \neq v_{2j}) \} )$$ 

 \item Contrainte pour que chaque case soit occupée par une pièce ou qu'elle soit menacée par au moins une pièce
    $$c_{2} = (y, \forall \hspace{0.1cm} 1 \leq i \leq n, \forall \hspace{0.1cm} 1 \leq j \leq n,  u1_{i,j} \cup u2_{i,j} \cup u3_{i,j} \cup u4_{i,j})$$
    
    avec : \begin{itemize}
    \item[•] $u1_{i,j} = \{(v_{1}, ..., v_{2m}) \in  D^{2m} \mid \exists k \in \{1, ..., m\}, v_{2k-1} = i \wedge v_{2k} = j\}$ \\
    \textit{La case \{i, j\} est occupée par une pièce}\\
   
   
    \item[•] $u2_{i,j} = \{(v_{1}, ..., v_{2m}) \in  D^{2m} \forall \mid \exists k \in \{1, ..., m\}, v_{2k-1} = i \vee v_{2k} = j\}$ \\
    \textit{La case \{i, j\} est menacée par au moins une tour}\\
    
    
    \item[•] $u3_{i,j} = \{(v_{1}, ..., v_{2m}) \in  D^{2m} \mid \exists k \in \{1, ..., m\}, p \in\{-n, n \}, ((v_{2k-1} = i+p \wedge v_{2k} = j+p) \vee ((v_{2k-1} = i-p \wedge v_{2k} = j+p))) \}$ \\
    \textit{La case \{i, j\} est menacée par au moins un fou}\\
    
    
    \item[•] $u4_{i,j} = \{(v_{1}, ..., v_{2m}) \in  D^{2m} \mid \exists k \in \{1, ..., m\}, p \in \{-2, 2 \}, l \in \{-1, 1\},
    ((v_{2k-1} = i+k \wedge v_{2k} = j+l)  \vee ((v_{2k-1} = i+l \wedge v_{2k} = j+k)))\}$ \\
    \textit{La case \{i, j\} est menacée par au moins un cavalier}\\
    
    \end{itemize}
    
\end{itemize}

\end{document}