\documentclass{article}

\usepackage[utf8]{inputenc}
\usepackage[T1]{fontenc}
\usepackage{hyperref}
\usepackage[french]{babel}
\usepackage{eurosym}

\title{INFO-F-302 Informatique Fondamentale\\ Projet : Problèmes de Satisfactions de Contraintes et Utilisation de l’Outil ChocoSolver}
\author{El Haman Abdeselam Abdeselam \and Minhas Prabhdeep}
\date{Année académique 2016-2017}

\begin{document}

\maketitle

\section{Problèmes d'échecs}
Soient n, k1, k2 et k3 des entiers positifs.
Dans le problème d'échecs, nous considérons un échiquier de dimensions n x n, ainsi que k1 tours, k2 fous et k3 cavaliers. 

Ces pièces ont une différente manière de se déplacer. Les tours peuvent aller verticalement ou horizontalement. Les fous, quant à eux, attaquent aux diagonales. Enfin, les cavaliers se déplacent en 'L', c’est-à-dire de deux cases dans une direction combinées avec une case perpendiculairement. \\

Nous pouvons compter deux problèmes dans les problèmes d'échecs, le problème d'indépendance ainsi que le problème de domination.
\begin{itemize}
\item Le problème d'indépendance consiste à déterminer s'il est possible d'assigner à chacune des pièce une position distincte sur l'échiquier de sorte qu'aucune pièce ne menace une autre pièce.
\item Le problème de domination consiste à déterminer s'il est possible d'assigner à chacune des pièces une position distincte sur l'échiquier de sorte que chaque case soit occupée ou menacée par au moins une pièce. \\
\end{itemize}

Afin d'exprimer ces deux problèmes par un CSP équivalent, nous avons commencé par écrire les variables, le domaine ainsi que les notations utilisées dans les contraintes pour ceux-ci.

La variable X est composée de toutes les coordonnées des pièces. Les tours sont notées $t_{i}$ et $t'_{i}$, les fous $f_{i}$ et $f'_{i}$ et finalement les cavaliers $h_{i}$ et $h'_{i}$.

Le domaine est compris entre 1 et n, n indiquant la taille de l'échiquier.

Pour les notations, nous avons la dimension de l'échiquier (n x n), ainsi que les notations pour représenter les différentes pièces (k1, k2, k3). 
De plus, m nous indique le nombre total de pièces que nous avons (k1 + k2 + k3). 
Enfin, y représente la suite de toutes les coordonnées des pièces, celles-ci sont alternées. Ainsi, les coordonnées x sont représentées sans les guillemets et les coordonnées y avec. Les 2 x k1 premières sont les tours, les 2 x k2 suivantes les fous et les 2 x k3 dernières les cavaliers.

\subsection{Question 1}
La première question nous demandait d'exprimer une instance quelconque du problème d'indépendance. Nous avons 4 contraintes pour ce problème.\\

La première contrainte nous indique que deux pièces ne peuvent pas se trouver au même endroit sur l'échiquier. 
Nous l'avons écrite comme ceci, pour un ensemble (v1, ..., v2m) $\in D^{2m}$, il existe un i et j, différent et compris entre 1 et m, le nombre de pièce, tel que ces deux pièces ne peuvent pas avoir les mêmes coordonnées $(v_{2i-1} \neq v_{2j-1}) \vee  (v_{2i} \neq v_{2j})$. \\

L'ensemble est jusque v2m car dans y nous avons le nombre de pièce multipliées par 2, pour avoir toutes les coordonnées. De la même manière, nous multiplions les indices i et j par deux, car nous les avons bornés jusque m, mais nous avons besoin de la seconde valeur dans y. Pour rappel, dans y, par exemple pour la première tour, nous avions mis $t_{1}$, suivit de $t'{1}$.\\

Pour les trois autres contraintes de ce problème, l'ensemble reste le même c'est seulement les conditions sur les deux pièces qui changent. 

La seconde contrainte se porte sur les tours, elle indique qu'aucune pièce ne peut se trouver sur l'horizontal ou la verticale d'une tour. Pour ceci, les bornes de i est entre 1 et k1, afin de ne prendre en compte que les tours, et j se trouve entre 1 et m.  Ces deux pièces ne peuvent être à la même colonne et à la même ligne, $(v_{2i-1} \neq v_{2j-1}) \wedge  (v_{2i} \neq v_{2j})$.\\

La troisième contrainte est sur la portée des fous, aucune pièce ne peut être sur les diagonales d'un fou. Pour cette contrainte, le i est entre k1+1 et k1+k2, afin de n'avoir que les fous, étant donné que les k1 représentent les tours nous faisons un saut du nombre de tour présents. Les bornes de j, par contre, restent les mêmes, de 1 à m. Dans cette contraintes, nous devons tenir en compte un autre indice qui définira les déplacements vers les diagonales, nous l'avons appelé k et celui-ci est compris entre \{-n, n\}.  Nous vérifions donc dans les 4 sens s'il n'y a pas d'autre pièces présentes,
$((v_{2i-1} \neq v_{2j-1}+k) \vee  (v_{2i} \neq v_{2j}+k)) \wedge  ((v_{2i-1} \neq v_{2j-1}-k) \vee  (v_{2i} \neq v_{2j}+k))$.\\

Enfin, pour la dernière contrainte sur les cavaliers, nous devons vérifier qu'aucune pièce ne se trouve sur les déplacements en L. Pour cela, la borne inférieure de i est maintenant k1+k2+1, nous sautons donc toutes les tours et les fous. Et donc la borne supérieure est bien le nombre total de pièce m. 
En plus de ceci, nous avons un k compris entre \{-2, 2\} et un l entre \{-1, 1\}, afin de pouvoir indiquer le déplacement vers une direction de deux case suivit d'un mouvement perpendiculaire, $((v_{2i-1} \neq v_{2j-1}+k) \vee  (v_{2i} \neq v_{2j}+l)) \wedge ((v_{2i-1} \neq v_{2j-1}+l) \vee  (v_{2i} \neq v_{2j}+k))$. 

\subsection{Question 2}

\subsection{Question 3}

\subsection{Question bonus}

\subsection{Question 4}

\section{Surveillance de musée}
\end{document}
